\part{Conclusiones}

\section{Conclusiones Generales y Desafíos}

El desarrollo de la presente práctica nos ha permitido poner en uso los conocimientos teóricos adquiridos no solo a la hora de evaluar el rendimiento y complejidad de algoritmos aplicados a grafos, sino además a llevar acabo el desarrollo de los mismo buscando el equilibrio entre resultados y coste.

Además mediante el análisis teórico y empírico a través de los resultados obtenidos hemos podido comprobar como en ocasiones los planteamientos teóricos pueden diferir de los resultados reales, siendo necesario un replanteo de la solución propuesta.\\

Por último, durante el desarrollo de la práctica hemos encontrado diversos desafíos siendo uno de los más relevantes el desarrollo la función generadora de grafos.

Ya que la generación de grafos afecta significativamente a como se evaluaran los diferentes algoritmos, en este caso se decidió que la función generadora de grafos recibiera como parámetro una probabilidad $p$, esta correspondería a la probabilidad de que entre dos nodos del grafo exista una arista, de esta forma se pueden evaluar los algoritmos para grafos con mayor o menor densidad de aristas fácilmente.


