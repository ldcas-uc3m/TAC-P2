\section{Estudio de PATH}\label{sec:path}


\subsection{Depth-First Search}\label{subsec:dfs}
El algoritmo DFS\supercite{hopcroft1983data} es un algoritmo para recorrer un grafo `en profundidad', y puede ser aplicado a nuestro problema $\mathrm{PATH}(u,v)$ recorriendo el grafo desde el nodo origen ($u$) y comprobando a cada paso si hemos llegado al nodo destino ($v$). En caso de que no lleguemos nunca, no hay camino.

\subsubsection*{Evaluación analítica del algoritmo}
Para analizar correctamente el algoritmo y sacar una cota asintótica superior, debemos de analizar el peor caso, % TODO: citation needed
el cual se da cuando el $u$ es el primer nodo, y el grafo está completamente conectado, excepto el nodo $v$, el cual está exclusivamente conectado al último nodo.



\subsection*{Evaluación empírica}

Para evaluar el comportamiento del algoritmo creado y comprobar si las conclusiones desarrolladas en la evaluación anaítica son correctas se lleva acabo una serie de tests, tomando la probabilidad de encontrar



\plotperformance{DFS}



\subsection{Floyd-Warshall}\label{subsec:fw}

\subsubsection*{Evaluación analítica del algoritmo}

Para analizar el coste computacional del algoritmo desarrollado, se puede dividir en tres secciones claramente diferenciadas para luego sumar sus costes y obtener un total, siendo dichas secciones: 
\begin{itemize}
    \item \textbf{Inicialización del Algoritmo}: Inicializar el algoritmo leyendo los parámetros e inicializar los dos nodos a computar el coste es de 7, ya que se trata únicamente de inicialización de variables y una comparación 'IF'.
    
    \item \textbf{Inicialización de matriz de distancias}: Esta sección del algoritmo cuenta con 2 bucles anidados que calculan la matriz de unos y ceros que indican que nodos se encuentran directamente conectados.
    \begin{itemize}
        \item \textbf{Bucle externo (u)}: Itera a través de cada nodo que actúa como un nodo intermedio en los caminos posibles, con un coste de $2n$ puesto que se modifica el valor de dos variables.
        \item \textbf{Bucle externo (v)}: Itera sobre todos los nodos posibles como nodo inicial, con un coste de n. En su interior el bucle consta de dos modificaciones de variables, y varias comparaciones 'IF' teniendo en el peor caso la comparación un coste de 3, por lo que el coste total del bucle sería de $5n$.
    \end{itemize}
    El costo de esta sección del algoritmo sería de un total de $10n^2$

    \item \textbf{Cálculo de la distancia entre nodos}: Itera sobre todos los nodos posibles como nodo final.
    \begin{itemize}
        \item \textbf{Bucle externo (k)}: Itera a través de cada nodo que actúa como un nodo intermedio en los caminos posibles, con un coste de n.
        \item \textbf{Bucle medio (i)}: Itera sobre todos los nodos posibles como nodo inicial, con un coste de $n$.
        \item \textbf{Bucle interno (j)}: Itera sobre todos los nodos posibles como nodo final, con un coste de n. Cuenta con una comparación 'IF' de coste 2 por lo que el coste total de este bucles sería de $2n$.
    \end{itemize}
    El costo de esta sección sería de un total de $2n^3$
    
\end{itemize}

Si se toma el costo de cada una de las tres secciones que componen el algoritmo, el costo total del algoritmo sería:
\begin{equation}
    T_{\mathrm{FW}}(n) = 2n^3 + 10n^2 + 7
\end{equation}

La complejidad temporal del algoritmo de Floyd-Warshall, viene dada por los tres bucles anidados donde se calcula la distancia entre nodos  siendo esta de O(n³).

\subsection*{Evaluación empírica}

Para evaluar el comportamiento del algoritmo creado y comprobar si las conclusiones desarrolladas en la evaluación analítica son correctas se lleva acabo una batería de pruebas con diferentes valores.

\plotperformance{FW}

Se observa que el costo computacional definido por T(n) se corresponde con los resultados obtenidos ya que se observa en la gráfica que el comportamiento que sigue el algoritmo se aproxima a ser cúbico.
