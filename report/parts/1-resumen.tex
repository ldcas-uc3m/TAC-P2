\part{Resumen}

La presente práctica tiene como objetivo el estudio de posibles algoritmos para resolución de problemas basados en grafos.

En la práctica se ha desarrollado un algoritmo para la generación de grafos sobre los que luego se aplicaran los algoritmos de resolución de problemas, dicho algoritmo toma como entrada el numero de nodos que tendrá el grafo y que probabilidad existirá de que los nodos se interconecten, de esta forma un grafo con una probabilidad baja tendrá poca densidad mientras que uno con alta probabilidad se acercará a ser completamente conectado.

Entre los algoritmos desarrollados se encuentran aquellos que buscan encontrar si existe un camino entre dos nodos dados como son DFS y Floyd-Warshall. 
Así como algoritmos que buscan resolver el problema de clique buscando dentro de un grafo, subgrafos totalmente conectados.

Para cada uno de los algoritmos propuestos se ha realizado una evaluación analítica describiendo su funcionamiento y como debería ser teóricamente su desempeño, así como una evaluación empírica llevando acabo pruebas para comprobar si el análisis propuesto se ajusta a los resultados reales en términos de complejidad y computabilidad.