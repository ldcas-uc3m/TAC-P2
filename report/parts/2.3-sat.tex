\section{Estudio K-SAT}
El problema SAT (problema de satisfacibilidad booleana)\supercite{gu1999algorithms} consiste en determinar si una fórmula booleana es \textit{satisfacible} (existe al menos un caso en el que es cierta).

Para resolver este problema, lo transformaremos en otro problema equivalente, CLIQUE, y resolveremos ese problema en su lugar.


\subsection*{Evaluación analítica del algoritmo}
Dado que el algoritmo consiste en transformar el problema en un grafo, y luego encontrar un K-CLIQUE dentro de ese grafo, aquí sólo nos centraremos en la transformación.

Asumiremos que el tamaño del problema, $n$ es igual al número de variables del SAT, $k$, por lo que $n=k$.

Podemos dividir el algoritmo en dos partes:
\begin{itemize}
  \item \textbf{Inicialización del algoritmo:} Coste de 7.
  \item \textbf{Obtener las variables:} Aquí se analiza la \textit{string} con la definición del problema y se extraen las variables. Se usa una búsqueda mediante \textit{regex}, la cual tiene una complejidad $O(m)$, siendo $m$ el tamaño de la \textit{string} de entrada. Como el tamaño de la entrada es proporcional al tamaño del problema, podemos asumir que tiene un coste $n$, además de otro coste $n$ para almacenarlo en una estructura de datos, dando un coste total de $2n$.
  \item \textbf{Generación del grafo:} Para generar el grafo se usa un bucle ejecutado $n^2$ veces, una por variable, en el cual se calcula el contrario de la variable (coste 4), y se conecta el nuevo nodo al resto de nodos anteriores. Esta conexión tiene un coste de $4\cdot\sum_{i=1}^{n}{i}=2n^2-2n$, dado que por cada nodo hay que conectarse a los que ya están. El total del coste de este paso es, por lo tanto, $6n^2-2n+7$.
\end{itemize}

Por lo tanto, el coste total de la transformación sería:
\begin{equation}
  T_{\mathrm{Transf-CLIQUE}}(k) = 6k^2 - 2k + 7
\end{equation}


\subsection*{Evaluación empírica}
Dado que solo se pidió analizar el caso 3-SAT, para todas las pruebas usaremos $k=3$.