\part{Apéndices}

\section{Instalación y ejecución}



\subsection*{Ejecución de los tests en Python}
Requiere Python\footurl{https://www.python.org/} 3.10+.
\begin{enumerate}
  \item Crea un \textit{virtual enviroment} en la carpeta \texttt{.venv/}:
  \begin{verbatim}
    python3 -m venv ./.venv
  \end{verbatim}
  \item Activa el entorno:
  \begin{itemize}
    \item Linux:
    \begin{verbatim}
source .venv/bin/activate
    \end{verbatim}
    \item Windows (PowerShell):
    \begin{verbatim}
& .\.venv\Scripts\Activate.ps1
    \end{verbatim}
  \end{itemize}
  \item Instala las dependencias:
  \begin{verbatim}
pip install -r requirements.txt
  \end{verbatim}
%   \item Compila el Turing Machine Simulator\supercite{tmsimulator} con GNU/Make\footurl{https://www.gnu.org/software/make/}:
%   \begin{verbatim}
% cd turing-machine-simulator
% make
% cd ..
%   \end{verbatim}
%   Si estás en Windows, te recomendamos instalar WSL2\footurl{https://learn.microsoft.com/es-es/windows/wsl/install} y ejecutar `en Linux', o con GCC a través de MinGW-W64\footurl{https://www.mingw-w64.org/} (puedes encontrar binarios ya compilados en \url{https://github.com/niXman/mingw-builds-binaries}), compilando mediante el comando \texttt{gcc} que puedes encontrar dentro de \texttt{turing-machine-simulator/\allowbreak Makefile}.
  \item Ejecuta el \textit{script} con:
  \begin{verbatim}
python3 src/test.py
  \end{verbatim}
\end{enumerate}
